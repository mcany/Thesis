\chapter{Glossary}
\begin{description}


\item[Broker] \hfill \\
 The Broker architectural pattern can be used to structure distributed software systems with decoupled components that interact by remote service invocations. A broker component is responsible for coordinating communication, such as forwarding requests, as well as for transmitting results and exceptions \cite{Buschmann}.
 

 
 \item[Layered Architecture] \hfill \\
  Partition the system into multiple interacting Layers with each layer
representing a specific responsibility or concern of relevance and
comprising all functionality that addresses that concern \cite{Buschmann}.
 
    
 
  \item[Plug-in Pattern] \hfill \\
  The Plug-in Pattern is a software pattern for extending the behavior of a class with a clean interface. Often behavior of classes is extended by class inheritance, where the derived class overwrites some of the virtual methods of the class. A problem with this solution is that it conflicts with implementation hiding. It also leads to situations where derived class become a gathering places of unrelated behavior extensions. Another way of extending the behavior of classes is by means of configuration. In this solution the class is given a set of primitives with which the user of the class can configure the behavior of the class. The problem with this solution is that the behavior extensions are limited to the configuration abilities that the root class provides, and to the chosen implementation. In a sense the root class contain unnecessary information. Both solutions make it difficult to dynamically switch between different behavior extensions.
  
 \item[E-textiles] \hfill \\
E-textiles, also known as smart garments, smart clothing, electronic textiles, smart textiles, or smart fabrics, are fabrics that enable digital components (including small computers), and electronics to be embedded in them.Smart textiles are fabrics that have been developed with new technologies that provide added value to the wearer. It is stated that "What makes smart fabrics revolutionary is that they have the ability to do many things that traditional fabrics cannot, including communicate, transform, conduct energy and even grow."
 
 \item[Repository Pattern] \hfill \\
The Repository Pattern is a common construct to avoid duplication of data access logic throughout our application. This includes direct access to a database, ORM, WCF dataservices, xml files and so on. The sole purpose of the repository is to hide the nitty gritty details of accessing the data. We can easily query the repository for data objects, without having to know how to provide things like a connection string. The repository behaves like a freely available in-memory data collection to which we can add, delete and update objects.


 \item[Event-driven architecture] \hfill \\
Event-driven architecture (EDA) is a software architecture pattern promoting the production, detection, consumption of, and reaction to events. An event can be defined as "a significant change in state". This architectural pattern may be applied by the design and implementation of applications and systems which transmit events among loosely coupled software components and services.

\end{description}


\label{chapter:ThisIsMyAppendixChapter}
