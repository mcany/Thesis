\chapter{Glossary}
\begin{description}


\item[Broker] \hfill \\
 The Broker architectural pattern can be used to structure distributed software systems with decoupled components that interact by remote service invocations. A broker component is responsible for coordinating communication, such as forwarding requests, as well as for transmitting results and exceptions \cite{Buschmann}.
 

 
 \item[Enterprise service bus ] \hfill \\
   Enterprise service bus (ESB) is a software architecture model used for designing and implementing the interaction and communication between mutually interacting software applications in service-oriented architecture.
    
 
  \item[Fixture] \hfill \\
   A fixture is an instance of a specific fixture type (see below). An instrumented space consists of a number of fixtures that affect the environmental conditions of the occupant.

  \item[Fixture type] \hfill \\
  A fixture type is the generalization of a fixture. Fixture types include:
  \begin{itemize}
  	\item Light
  	\item Addressable plug
  	\item Window blinds
  	\item Window/door sensor
  \end{itemize}

 \item[Smart environment] \hfill \\
A highly integrated computing and sensory environment that effectively reasons about the physical and user context of the space to transparently act on human desires .
\begin{itemize}
\item Highly integrated - an environment that is saturated with actuators and sensors that are fully integrated with wireless networks
\item Effectively reason - a pseudo-intelligent reasoning mechanism for the environment as a whole, not just to an individual device or component 
\item User context - an individual profiles, policies, current location and mobility status
\item Transparently act - an environment that is responsive to humans and supports their mobility without requiring user interaction
\end{itemize}

%(http://www.opengroup.org/soa/source-book/soa/soa.htm#soa_definition) \\

 
 \item[Service Oriented Architecture] \hfill \\
 Service-oriented architecture (SOA) is a software design and software architecture design pattern based on discrete pieces of software that provide application functionality as services, known as Service-orientation. A service is a self-contained logical representation of a repeatable function or activity .   
 
  

 \item[Web service] \hfill \\
A Web service is a software system designed to support interoperable machine-to-machine interaction over a network. It has an interface described in a machine-processable format (specifically WSDL). Other systems interact with the Web service in a manner prescribed by its description using SOAP-messages, typically conveyed using HTTP with an XML serialization in conjunction with other Web-related standards .

\end{description}
\label{chapter:ThisIsMyAppendixChapter}
