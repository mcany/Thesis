\chapter{Introduction}
\label{chapter:Introduction}



	E-textiles, also known as smart garments, smart clothing, electronic textiles, smart textiles, or smart fabrics, are fabrics that contain embedded digital components (including small computers) and electronics \cite{1}. Simply put, e-textiles are fabrics with electrical characteristics. These electronic devices can be worn on clothes, stored in pockets, held in the hand, or even strapped to the head. E-textiles provide an effective system to store or connect computing elements \cite{12}. 
\\ 
Sensors and low-power processors are now small enough and inexpensive enough to allow the deployment of wireless ad hoc networks for various applications \cite{11}. Recent technological developments have enabled miniaturization; smaller, lower-priced, and lighter chips; wireless sensor networks (WSN); and fiber level research. Thus, e-textiles have emerged, and the research focus has shifted from wearable devices to e-textiles. As in classical electronics, the construction of electronic capabilities in textile fibers requires the use of conducting and semiconducting materials such as conductive textiles \cite{1}. One of the most important issues of e-textiles is that the fibers should be made in a way that they can be insulated and completely washable, so they can be successfully used in clothing manufacture. Fortunately, as a result of fiber level research, a new class of electronic materials has emerged that helps to create e-textile products that acts like a normal clothes. That is why the new class of electronic materials is more suitable for e-textiles and solves daily use problems. It is a class of organic electronics materials that can be conducting and semiconducting, and they can be designed as inks and plastics. As a result of fiber level development, woven circuits are now possible to be produced \cite{4}. These woven circuits enable the integration of electronic components in woven e-textiles at the yarn level. Furthermore, in connection to the continuing development of wireless sensor networks, the authors in \cite{9} state that the excitement about this technology is motivated by the several benefits associated with the long-term monitoring, low cost, rapid deployment, self-organization, and flexibility features of WSN. These different technological developments make it possible to create e-textile products with electronic devices that can be worn, bent, washed, and carried without attaching any additional devices. For example, rather than attaching electronic devices to the body with strips, the strips are now the electronic components. Thus, because of all these technological advancements, electronic components can now be integrated into clothing. 
\\
The field of e-textiles can be divided into two main categories:
\begin{itemize}

\item E-textiles with classical electronic devices such as conductors, integrated circuits, LEDs and conventional batteries embedded into garments.
\item E-textiles with electronics integrated directly into the textile substrates.

\end{itemize}

Since textiles play an essential role in human existence, e-textiles possess the potential to be used in many aspects of our lives. Wearable textile products can be utilized in many different areas, including sports, fashion, medicine, rehabilitation, context awareness systems, and social media integration \cite{12}. Representative applications are sports training data acquisition, health monitoring of vital signs, monitoring personnel handling of hazardous materials, tracking the position and the status of soldiers in the field, and monitoring pilot or truck driver fatigue. Furthermore, in the area of rehabilitation as stated in \cite{9}, rehabilitation requires continuous supervision during long-term rehabilitation therapy. This continuous supervision increases the workload for physical therapists and medical staff, and it is very expensive for the patients. As a result of such challenges, new sensor-based e-textile solutions have arisen from the need to develop effective, low-cost and easy to use rehabilitation supervision systems. Using e-textile solutions dramatically reduces the cost and size of systems and opens new opportunities \cite{9}. 
\\
Since the field of e-textiles has subcategories, the word e-textile is used to identify a variety of concepts. These concepts include fiber level development, textile sensors, and application development for smart garments (health/rehabilitation, fitness, fashion, maintenance). This thesis focuses on the application level development for e-textiles. In this paper, I present the design and implementation of a framework for rapid development of e-textile applications. 
\\

\section{Problem Statement} \label{problemStatement}

	Wearable devices and the smart garments has created a significant research interest over the last decades \cite{2}. This research interest has occurred by the commercial interest. As a result it can be said that smart garments are highly popular among not only researchers but also end users. Despite this interest by both the developers and the end users, only a few developments have exceeded a prototypical level \cite{2}. Moreover, problems still exist that prevent an efficient level of mass production. 
	\\	
	
The basic materials used to produce e-textiles, which are conductive threads and fabrics, have been around for over 1,000 years. However, the e-textile market is still in its infancy, and the quantity of e-textiles being produced is limited. One main reason for and resulting criticism of this limited development and production of e-textiles is the slow and complicated process currently involved. 
\\
	
	
	The process of development an e-textile application is actually slower and complicated when it is compared to alternative mobile or web application development. One of the reasons for the this slow and complicated process is that there are few good high level sources of information or research data to give a practical insight into making e-textile development easier and quicker. The development must go through the following steps: signal receiving, synchronizing, signal filtering, and feature extraction. These essential steps generally need to be done in order to develop an e-textile application. However, when new developers are interested in working in this area, it is hard for them to find solid resources to aid in the development of their applications. Thus, because new developers may not have a comprehensive grasp of the essential steps in e-textile development, they often do not know where to start. 
	\\
	
	 Another reason is that the knowledge that is required to develop e-textiles is complex. The development of e-textiles includes sensor interfaces, synchronization of the sensor signals, and signal filtering \cite{8}. For beginners in the field, it is first required to experiment with the signals, to then review some examples, and finally to use an interactive development tool in order to get some idea about the concept. The IDEs and libraries that are used for e-textile development will be explained in detail in the Related Work section. They focus on either students that are not professional developers or experienced developers that have previous experience with e-textiles. These IDEs and libraries do not actually support complex application development since they lack testing, simulation, visualization, and predefined algorithms for machine learning or filtering for a real e-textile devices. The problem here is the lack of supporting entire process in development of an e-textile application. Part of the steps can be achieved by using the mentioned IDEs and libraries partially if the other steps are achieved manually. In order to realize all the steps in one environment, all of these procedures should be implemented manually. So that this creates an expectation that users have a deep knowledge about e-textile development, and so they do not really support newcomers to the field.  \\	
	
There is a jungle of libraries and protocols and these libraries and protocols are either vendor specific or operate on the hardware level; for example, the developers must be able to handle the hardware I/0. In this situation, the developers need to start from the beginning by checking hardware level development. This low-level development is another reason for why overall e-textile development is such a slow process. Since there is no good abstractions for the development, users need to work on the hardware elements and develop them manually. This situation creates a time issue for the developers and they felt rushed the development and the problems. Since this low-level of development requires specific information about the device, most developers lack this kind of information, and it is hard to obtain this knowledge quickly and efficiently. Hence, that is the reason development is separated from device in other areas like communication, such as with servers and mobile devices. This device application separation is not available in e-textile science at this point in time.  	\\
	
	Furthermore most developers are not experts in signal processing or activity recognition but want to realize their application concepts using a high level API \cite{2}. Niels Henze et al argue that there is a lack of tool support for application developers and no enough defined APIs within the software and hardware stack that allows developing useful smart garment applications \cite{2}. The hardware that is used to produce e-textile products, like raspberry-pi, Arduino or Intel, does no support for higher level development, there is not enough protocols and libraries to interact with hardware components. The problem is there is no application layer in the e-textile development yet. (Fig. 1.1). As a result, all the communications with hardware, such as reading the data from sensors or writing to actuators, are needed to be done by the developers. This situation makes building more complex systems even harder. Like in development of mobile or desktop application, in the development of e-textile also requires for APIs that support more complex commonly used functions or functionality such as data mining and machine learning. Unfortunately in the current situation there is no tool support for application developers and no defined APIs within the software \cite{2}. The development is on specific devices that the libraries or protocols are designed for. This makes an application is specific to that devices or vendors. All the signal-processing algorithms, signal synchronization or filtering methods are developed specific to the used device because all the input output handling are specifically designed according to the used hardware. Thus the knowledge about developing on hardware is required from the developers. Since expecting a good knowledge on hardware development is not realistic, there are some researches are still going on about creating a top development level \cite{2}, \cite{4}, \cite{5}, \cite{8}, \cite{9}. \\

	Most of the development environments, which are used to test different system variants, are not suitable for actually running applications \cite{8}. These environments are dependent upon custom engines or libraries. For this reason, in order to test different systems, implementing an actual e-textile application for specific device is necessary. In order to do this, developing selected algorithms in an appropriate programming language and then distributing them to specific devices are necessary. Relevant issues in this process include sensor interfaces, synchronization of the signals, and optimizing for specific devices. Although the applications are similar, when the developers want to test them on other devices, since all the implementations are done for a specific device, the same development has to be done from the beginning for any different hardware.  \\
	

	E-textile development is similar to distributed systems. It is hard to communicate and synchronize between different sensors. Moreover, as stated before, it is hard to simulate the application before actual implementation; therefore, it is not easy to test a system. Developers may not know which algorithms to use in their application. For example, just for filtering alone, there are more than twenty-four algorithms that can be used for smoothing the signals \cite{10}. In order to find the best one, it requires several phases of testing. Also, the algorithms change according to the data and the sensor. Since there is no easy way to simulate the data and test them with different algorithms to find the best ones, the developers need to implement every algorithm that they want to compare, in order to find the suitable one. The developers need to collect the data again, apply the algorithm, see the results, and compare them. This cycle needs to be done for different algorithms to compare the results and find the appropriate ones. They cannot change the algorithm in the apply algorithm phase. If they want to test a different algorithm, they need to start the process from the beginning. This is a time-consuming process when the system is more complex. 


\insertfigure{images/mcan/Introduction/layered.png}{Presented Development Layer}{UserFixtures}{0.30}



\section{Solution}

According to problems that are described in section \ref{problemStatement}, an interactive e-textile development environment, so called TextIt, is proposed as a solution in order to increase the speed of development of e-textile application. As a result more developers may join to production of e-textile application without any previous knowledge requirement about e-textiles. \\ 


The solution of development environment that is proposed is to create some kind of an e-textile playground for the developers. In that development environment with the playground feature developers can see how a data looks like, can use the pre-defined components and see the affects, let the system handle device communication and input output handling. Developers may get some sample data or receive the real data from e-textile products and have predefined filtering, feature extracting and machine learning components, which are required to process data, so that they can try their own application ideas by using these components. These filtering, feature extracting and machine learning components have to be extracted according to current market needs. In the e-textile applications there are some common steps that are used several times like Bluetooth communication, signal processing, filtering, machine learning and etc. Since these steps are common and there are not some standardized solutions for the development, the environment should support pre-defined solutions for those common steps. Implementing those components for every application is a wasted effort and money especially they do not change over time. That is why the proposed solution comes with a pre-defined algorithms that are used in the development of an e-textile application. Moreover since the developers are required to test and experiment with different algorithms before the actual implementation, the development environment can solve this problem by providing the algorithms that the developers need and can show the result of different algorithms. By this approach the developers can see the effects of the algorithms and choose the best one for their applications and use it for actual implementation. This support for pre-defined components helps developers to explore more and be more free on choosing algorithm. With this approach rather worrying about how to implement the algorithms, the developers can spend more time on designing phase and be more imaginative and think about more complex applications. Focusing on the actual application and thinking about what can be done rather than how can be done can help e-textile to be more innovative and can address the problems of the market. \\

As a result, to realize the solution of a development environment for e-textiles, needed APIs, pre-defined algorithms and components are required to be identified. The purpose to collect those requirements under one environment is to support developers to easily access the resources specifically for e-textile development. The aims of this e-textile environment, so called TextIt \ in this thesis, are to speed up the development of e-textiles, make it available for many developers that have different knowledge background and make the applications hardware independent to extend its usage area.
