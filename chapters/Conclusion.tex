\chapter{Conclusion}

In the following chapter, a discussion about the design of TextIt is given, along with the future work and summary of what has been done during the process of writing this thesis. The discussion section describes the advantages and disadvantages of the framework, in addition to the suggestions given on how the system can be improved to have a better user experience. Future work would increase the usability of the system, and make the development of e-textile application easier and faster. 


\section{Discussion}
TextIt is decomposed into seven components (Gathering Controller, Etextile Communication Selection, View Controller, Interactex Server Controller, BLE-Firmata Controller, JavaScript Runner, File), where every component is responsible for one aspect of the system (e-textile communication, user input handling, data receiving and manipulating).
\\ 

	An advantage of this design is that the implementation of every component can be done independently, as long as the interface of the component does not change. The components are implemented using the design-by-contract approach. The key concept of design-by-contract is viewing the relationship between a class and its clients as a formal agreement, expressing each party's rights and obligations. The existence of a contract also means that every component can be developed by a different person or a team, without knowing anything about the functioning of other components. \\
	
Another advantage of this design is that the components and the user code can easily be shared with other devices and applications (like Interactex) without the need for changing the components themselves. The advantage that emerges from the ease of distribution and the fact that the interpreted language context are stateless, is that all the components and user code can be replicated and deployed on multiple applications and devices.\\

	A disadvantage of this design is that the user code is executed at the background in another thread in every 2 seconds. User input or data traffic load influences the response time of the system. The bigger the data load is or the more the user input is, the longer the response time of the system will be. Currently, for the use cases the response time of the system is around 1 second which is considered as a normal waiting time. Another main reason for the possible late response time is that for the simplicity of the thesis in every user change the whole user input is executed from the beginning except the visualization codes. If a more complex system is designed then because of the computation time and the re-execution of the user input the response time will be increased. \\ 
 
 The response time is expected to decrease by understanding the change in the user code and execute the rest of the code instead of executing the whole code. Using this approach, the overload on the background thread and the computation load will be minimized. For the proposed solution there is no architecture change is required. There might be the case that the new components can be added but that does not require a change in the architecture. The proposed two architecture can continue to be used.  

\insertfigure{images/mcan/SystemDesign/ComponentDiagram.jpg}{Middleware Architecture of TextIt}{Middleware}{1.00}

		\insertfigure{images/mcan/SystemDesign/LayeredDiagram.jpg}{Layered Architecture of TextIt}{Layered}{0.90}




\section{Future Work}

The emphasis of the future work should be on increasing the usability of the system and user experience of the e-textile developers. The usability of the system and comfort of the inhabitants can be increased by the following activities:

\begin{itemize}
\item implementing the missing functionality
\item adding more pre-defined algorithms
\item adding more communication protocols to support more e-textile devices
\item full integration with Interactex or with other applications
\item adding data visualization with the reference video
\end{itemize}

\subsection{Implementing the Missing Functionality}

Adding, removing and changing the name or the folder of the files from the system is not fully functional. In order to give full experience of the development environment these functionality should be implemented. \\

Also as discussed in Discussion section, instead of executing the whole code in every change, the user change should be identified and execute the code respectfully. This should increase the efficiency and decrease the response time. So that the behavior would be more convenient.
	
\subsection{Adding More Pre-defined Algorithms}

Right now there are three different main algorithm sections are included in the system (Filter, Feature Extraction and Machine Learning). These are the ones that are extracted from the use cases. However this sections can later be increased with new groups. So that the developer can find whatever he needs in the system and does not have to think about how to implement it. \\

Moreover even for the three algorithm sections, not every algorithms are implemented. For example for filtering low-pass filter, high-pass filter and RC filter are implemented. There can be find many other filtering algorithms for the data. Same applies for the feature extraction and machine learning algorithms. For machine learning section since the uses cases do not need and for the time pressure no sample algorithms are added. For the feature extraction section; peak detection, standard deviation, median, mean and fast Fourier transform algorithms are added. 


\subsection{New Communication Protocols for New Devices}
As stated in the definition of an e-textile development environment in the introduction of this thesis, the system has to be able to communicate with different e-textile devices. In order to do this BLE and Firmata protocols are used.	\\ 

For the expansion of supported e-textile devices, the addition of new communication types is inevitable. By implementing more communication protocols, the more e-textile device can be connected to the system. This gives developers freedom when they choose a device. \\

Addition of every new communication type requires adding new parsing method if the protocols are using different representation for the data.

\subsection{Full Integration with Interactex or with Other Applications}
In this current situation TextIt can be able to send a sample code to Interactex as a custom component. As the system gets improved and added new components the counter components should also be added to Interactex. Moreover, the code should be cleaned from custom code that are used to visualize or to connect to e-textile device before sending to the Interactex. The custom visualization code or connection code are defined in the system so that they can be identified and cleaned from the code easily. All the communication and uploading features are already added to the system the missing part is the cleaning the code from custom methods before uploading it.\\ 

The integration with Interactex can only be an example that the system can be used with other e-textile development environments if needed. The new interactions can be added to the system independently thanks to architecture.

\subsection{Data Visualization with the Reference Video}

This reference video section can be also found in the mock up design. The purpose of this section is that when the data is recorded from an e-textile device, the users also record the activity. When they visualize the data on the TextIt, they can also give a video as a reference so that which point corresponds to which part of the activity. To give an example in one of use case called "One Leg Hop", the time of the jump is required. In order to calculate the duration of the jump the staring point of the jump and the landing should be identified. This is not an easy task especially if what peaks mean in the data. To give a user better insight about the data, the user can upload a video and the data is synced with the video. After that the user can see which point corresponds to which action in the video. With that approach the full information about the data is given to users. The area for that window is left empty for future development.

\section{Summary}

During the course of writing this thesis, an integrated development environment TextIt for developing an application for e-textile devices was designed and implemented. The emphasis was placed on designing and implementing an environment which would provide a data gathering, visualization and manipulation with help of pre-defined algorithms. Also, the emphasis was on enabling communication with different e-textilde devices and connection with other independent environment such as Interactex. The TextIt is made up of six different components, with every component performing a different task in the system.  \\ 

The TextIt enables the user to develop an application for e-textiles on an Application Layer with the abstractions. Besides that, the TextIt helps new beginners with a playground feature to make them understand the data better and use pre-defined algorithms for easy development. The systems helps experienced developers to build an application with an easy and efficient way without any vendor specification. \\

TextIt does not yet support the full feature of an development environment with some missing feature. Nevertheless, the design of the TextIt supports development of components that are discussed in the \textit{Future Work} section that will add this kind of functionality to the environment. The data gathered by the TextIt along with the ability provided by the TextIt to control the data in the environment, enables the easy and fast development in the e-textile application areas. As a result of this research, new applications, which would add more value and interest to e-textile area, can be achieved with the help of TextIt.