% Abstract for the TUM report document
% Included by MAIN.TEX


\clearemptydoublepage
\phantomsection
\addcontentsline{toc}{chapter}{Abstract}	

\vspace*{2cm}
\begin{center}
{\Large \bf Abstract}
\end{center}
\vspace{1cm}


\def\sife {Smart Environment Integration Framework}
\def\TI{TextIt}

\def\seif {Smart Environment Integration Framework}

Electronic textiles (E-Textiles) are a good container for wearable technology and wearable computing specifically to produce activity recognition systems. They enable to use digital components and electronics to embed them into textiles. An activity recognition system built with e-textile can be used in a variety of ares like health monitoring, military and fashion. \\

Since electronic textile is a new concept, it suffers from technology related problems that have not been solved sufficiently; problems such as time consuming data collection, low-level development, unable to test different approaches in an efficient way and no good platform to experience. As a consequence, gathering, observing and testing information from e-textiles and wearable devices poses a challenge, and the development a product is slow and tedious. \\

In this thesis, a complete integrated development environment (IDE) to develop an e-textile product with gathering the data, observing and processing it and testing different algorithms has been proposed. The idea is to combine the playground feature for inexperienced developers and the IDE for e-textiles and wearable devices. User will not need any additional application to get the data from e-textile or wearable devices or any other applications (like WEKA or Matlab) to test and process the signal. \\

The focus of this thesis is on the design and the development of a framework for e-textile development. The framework is a combination of a playground and a conventional IDE. Text based development framework called TextIt that is used for gathering, observing, testing for the e-textile application. The system is developed with playground support to make it easy to enter this area as an inexperienced developer. As the system is able to show the data, have pre-defined algorithms for signal processing and visualize the different test algorithms and compare the results. 

%One page!
%What is the content of your thesis? What are the results, what have you achieved?

%Common mistake:
%This is NOT an appetizer to read your whole thesis. The most interesting things should already be in %the abstract.