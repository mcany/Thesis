\clearemptydoublepage

\phantomsection
\addcontentsline{toc}{chapter}{Outline of the Thesis}


\begin{center}
	\huge{Outline of the Thesis} 
\end{center}
\vspace{4mm}
%--------------------------------------------------------------------




\noindent {\scshape Chapter 1: Introduction}  \vspace{1mm}



\noindent The introduction chapter starts with the definition of an e-textile, and a motivation on how it can improve human lives. This chapter describes the challenges that are faced when creating an e-textile product, and proposes a solution for overcoming these challenges.\\

\noindent {\scshape Chapter 2: Requirements Elicitation}  \vspace{1mm}

\noindent  This chapter describes the requirements elicitation which includes the definition of scenarios that drive the development of the TextIT. Based on these scenarios, non-functional requirements are proposed and a functional model with use cases and functional requirements is elaborated. \\

\noindent {\scshape Chapter 3: Analysis}  \vspace{1mm}

\noindent  This chapter provides a description of an analysis model based on the requirements. The model is created to formalize the objects and information that exist in the domain of the e-textile environment. It identifies the entity, boundary and control objects of the TextIt. The created object and sequence diagrams provide a better understanding of the system requirements. As a part of the analysis object model, abstract factory were used for modeling a e-textile environment. By using aa abstract factory pattern tackles a problem of modeling an e-textile communication with multiple messaging types that may execute different set of commands, while each device can use a different communication protocol. \\

\noindent {\scshape Chapter 4: System Design}  \vspace{1mm}

\noindent  This chapter gives a description of the system design model which contains the strategies and practices for designing the TextIt. As a part of the design model, the TextIt is decomposed into smaller subsystems, and the hardware/software mapping is shown. The broker pattern is used for decoupling clients from communication tasks. The broker component is decomposed into sub-components, and the role of each sub-component is shortly described. Finally, persistent data management, software control and boundary conditions are discussed. \\

\noindent {\scshape Chapter 5: Object Design}  \vspace{1mm}

\noindent In this chapter, a description of the components of the broker for controlling the e-textile is given. The interfaces of the components are described in detail. \\

\noindent {\scshape Chapter 6: Conclusion}  \vspace{1mm}

\noindent  This chapter gives a summary of what was done during the course of the thesis. Afterwards, the results are reviewed critically and visions for future work are pointed out. \\

